% ------------------------------------------------------------------------
% PROPRIEDADES DO DOCUMENTO
% ------------------------------------------------------------------------
\documentclass[12pt,
openright, 
oneside, %
%twoside, %TCC: Se seu texto tem mais de 100 páginas, descomente esta linha e comente a anterior
a4paper,    %
%english,   %
brazil]{facom-ufu-abntex2}

% ------------------------------------------------------------------------
% PACOTES
% ------------------------------------------------------------------------
\usepackage{amsmath}
\usepackage{icomma}
\usepackage[portuguese,ruled]{algorithm2e}
\usepackage{hyperref}
\hypersetup{
    colorlinks = false,
    hidelinks = true
}
\selectlanguage{portuguese}

% ---
% Configurações do pacote backref
% Usado sem a opção hyperpageref de backref
\renewcommand{\backrefpagesname}{}
% Texto padrão antes do número das páginas
\renewcommand{\backref}{}
% Define os textos da citação
\renewcommand*{\backrefalt}[4]{}%
% ---
% ------------------------------------------------------------------------
% INFO para CAPA e FOLHA DE ROSTO 
% ------------------------------------------------------------------------
\titulo{XXX Título do Trabalho XXX} %TCC

\autor{XXX Nome do Autor XXX} %TCC
\data{XXX Ano XXX}

\orientador{XXX Nome completo do orientador XXX} %TCC
%\coorientador{Nome completo do orientador caso tenha} %TCC

\begin{document}
\frenchspacing 

% ----------------------------------------------------------
% ELEMENTOS PRÉ-TEXTUAIS
% ----------------------------------------------------------
%\pretextual
\imprimircapa
\imprimirfolhaderosto


% --- Inserir folha de aprovação --- %
\begin{folhadeaprovacao}

  \begin{center}
    {\ABNTEXchapterfont\large\imprimirautor}

    \vspace*{\fill}\vspace*{\fill}
    {\ABNTEXchapterfont\bfseries\Large\imprimirtitulo}
    \vspace*{\fill}
    
    \hspace{.45\textwidth}
    \begin{minipage}{.5\textwidth}
        \imprimirpreambulo
    \end{minipage}%
    \vspace*{\fill}
   \end{center}
    
   Trabalho aprovado. \imprimirlocal, XXX DIA de MÊS de ANO XXX: % dia da apresentação

   \assinatura{\textbf{\imprimirorientador} \\ Orientador}  
   \assinatura{\textbf{XXX MEMBRO DA BANCA 1 XXX}}% \\ Convidado 1} %TCC:
   \assinatura{\textbf{XXX MEMBRO DA BANCA 2 XXX}}% \\ Convidado 2} %TCC:
   %\assinatura{\textbf{Professor} \\ Convidado 3}
   %\assinatura{\textbf{Professor} \\ Convidado 4}
      
   \begin{center}
    \vspace*{0.5cm}
    {\large\imprimirlocal}
    \par
    {\large\imprimirdata}
    \vspace*{1cm}
  \end{center}
  
\end{folhadeaprovacao}
% \includepdf{folhadeaprovacao_final.pdf} % TCC: depois de aprovado o trabalho, descomente esta linha e comente a anterior para incluir o scan da folha de aprovação.

%% OBS.: as seções dedicatória, agradecimento e epígrafe não são obrigatórias. Só as mantenha se achar pertinente.

% --- Dedicatória --- %
\imprimirdedicatoria{XXX Dedicatória XXX}

% --- Agradecimentos --- %
\imprimiragradecimentos{XXX Agradecimentos XXX}

% --- Epígrafe --- %
% \imprimirepigrafe{Insere alguma citação que ache conveniente caso queira ...}
	
% --- Resumo em português --- %
\begin{resumo} %TCC:
XXX Resumo XXX
 
 \vspace{\onelineskip}
 \noindent
 \textbf{Palavras-chave}: XXX Palavras, chave XXX. %TCC:
\end{resumo}

\renewcommand{\resumoname}{Abstract}
\begin{resumo}\itshape %TCC:
    XXX Resumo em inglês XXX

 \vspace{\onelineskip}
 \noindent
 {\normalfont \textbf{ Palavras-chave}}: XXX Palavras, chave, em, inglês XXX. %TCC:
 
\end{resumo}

% --- lista de ilustrações --- %
\listailustracoes

% --- lista de tabelas --- %
\listatabelas

% --- lista de abreviaturas e siglas --- %
\begin{siglas} %TCC:
	
	% 0
	
	% 1
	
	% 2
	
	% 3
	
	% 4
	
	% 5
	
	% 6
	
	% 7
	
	% 8
	
	% 9
	
	% A

	% B
	
	% C
	
	% D
	
	% E
	
	% F
	
	% G
	\item[\textit{GUI}] \textit{Graphical User Interface}
	\item[\textit{GPU}] \textit{Graphics Processing Unit}
	% H
	
	% I
	\item[IA] Inteligência Artificial
	
	% J
	
	% K
	
	% L
	
	% M
	\item[\textit{MSE}] \textit{Mean Squared Error}
	% N
	
	% O
	
	% P
	
	% Q
	
	% R
	\item[ReLU] \textit{Rectifier Linear Unit}
	\item[RNA] Redes Neurais Artificial
	% S
	
	% T
	\item[\textit{TPU}] \textit{Tensor Processing Unit}
	
	% U
	
	% W
	
	% V
	
	% X
	
	% Y
	
	% Z

\end{siglas}




% --- lista de símbolos --- %
\begin{simbolos}
	% 0
	
	% 1
	
	% 2
	
	% 3
	
	% 4
	
	% 5
	
	% 6
	
	% 7
	
	% 8
	
	% 9
	
	% A
	\item[$\alpha$] Taxa de aprendizado
	
	% B
	\item[$b$] Viés
	
	% C
	
	% D
	
	% E
	\item[$\epsilon$] Deformação
	
	% F
	
	% G
	
	% H
	
	% I
	
	% J
	\item[$J$] Índice de desempenho
	
	% K
	\item[$K$] Coeficiente de resistência
	% L
	
	% M
	
	% N
	\item[$n$] Expoente de encruamento
	\item[$N_b$] Tamanho de lote
	\item[$\nu$] Coeficiente de Poisson
	% O
	
	% P
	
	% Q
	
	% R
	
	% S
	\item[$\sigma$] Tensão
	\item[$\sigma_y$] Limite de escoamento
	\item[$\sigma_i$] Variância dos valores do parâmetro de entrada $i$
	% T
	
	% U
	\item[$u$] Potencial de ativação
	
	% W
	\item[$w_i$] Pesos do perceptron/RNA
	
	% V
	
	% X
	\item[$x_i$] Entradas do perceptron/RNA
	\item[$\overline{X_i}$] Média dos valores do parâmetro de entrada $i$
	
	% Y
	\item[$y$] Saída do perceptron
	
	% Z
	\item[$z_i$] Entrada normalizada 	
\end{simbolos} % caso não existe símbolos no trabalho, comente esta linha

% --- sumario --- %
\sumario


% ----------------------------------------------------------
% ELEMENTOS TEXTUAIS
% ----------------------------------------------------------
\textual

% --- Introdução --- %
% ----------------------------------------------------------
% Introdução
% ----------------------------------------------------------
\chapter[Introdução]{Introdução}

Escreva a introdução aqui.

% ---  Fundamentação teórica --- %
% \item---------------------------------------------------------
% Indentação
% ----------------------------------------------------------
\chapter[Fundamentação Teórica]{Fundamentação Teórica}



% --- Desenvolvimento --- %
% ----------------------------------------------------------
% Redes Neurais Artificiais
% ----------------------------------------------------------
\chapter[Desenvolvimento]{Desenvolvimento}


% --- Resultados --- %
% ----------------------------------------------------------
% Resultados
% ----------------------------------------------------------
\chapter[Resultados]{Resultados}


% --- Conclusão --- %
% ----------------------------------------------------------
% Conclusão
% ----------------------------------------------------------
\chapter[Conclusões e sugestões de trabalhos futuros]{Conclusões e sugestões de trabalhos futuros}



% ----------------------------------------------------------
% ELEMENTOS PÓS-TEXTUAIS
% ----------------------------------------------------------
\postextual

% --- Referências bibliográficas --- %
\bibliography{blibiografia}

% --- Apêndices --- %
% só mantenha se for pertinente.
% \begin{apendicesenv}
% \partapendices % Imprime uma página indicando o início dos apêndices

% % --- Apendice 1--- %
% % ----------------------------------------------------------
% Apendice 1
% ----------------------------------------------------------
\chapter{Coisas que fiz e que achei interessante mas não tanto para entrar no corpo do texto}

\lipsum[50]

% \end{apendicesenv}

% --- Anexos --- %
% so mantenha se pertinente.
% \begin{anexosenv}
% \partanexos % Imprime uma página indicando o início dos anexos

% --- Anexo 1 --- %
% % ----------------------------------------------------------
% Anexo 1
% ----------------------------------------------------------
\chapter{Coisas que eu não fiz mas que achei interessante o suficiente para colocar aqui}


\lipsum[31]

% \end{anexosenv}


% ------------------------------------------------------------------------
% FIM DO DOCUMENTO
% ------------------------------------------------------------------------
\printindex
\end{document}